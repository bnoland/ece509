\message{ !name(hw1.tex)}% TODO: AUCTeX insists on some annoying indentation conventions by
% default. Change this.

\documentclass[letterpaper,12pt]{article}
\usepackage[utf8]{inputenc}
\usepackage[top=1.25in, bottom=1.25in, left=1in, right=1in]{geometry}
\usepackage{amsmath}
\usepackage{amsfonts}
\usepackage{enumitem}

\setenumerate{parsep=0em, listparindent=\parindent}

\DeclareMathOperator{\Tr}{Tr}

\title{Homework 1}
\author{Benjamin Noland}
\date{}

\begin{document}

\message{ !name(hw1.tex) !offset(-3) }


\maketitle

\begin{enumerate}
\item Let $C \subseteq \mathbb{R}^n$ be a convex set, and let
  $L \subseteq \mathbb{R}^n$ be a line. Then $L$ is also a convex
  set. Since set intersection preserves convexity, $L \cap C$ is
  therefore a convex set.

  Conversely, let $C \subset \mathbb{R}^n$ be an arbitrary set, and
  suppose that for every line $L \subseteq \mathbb{R}^n$, $L \cap C$
  is convex. Let $x_1, x_2 \in C$ and $0 \leq \theta \leq 1$. Let
  \begin{equation*}
    L = \{\alpha x_1 + (1 - \alpha) x_2 : \alpha \in \mathbb{R}\}.
  \end{equation*}
  Then $L \cap C$ is convex by assumption. Thus, since
  $x_1, x_2 \in L \cap C$,
  $\theta x_1 + (1 - \theta) x_2 \in L \cap C$, and so
  $\theta x_1 + (1 - \theta) x_2 \in C$ in particular. Thus $C$ is
  convex.

  \bigskip

  The corresponding result for affine sets requires a short lemma: if
  $S_1, S_2 \subseteq \mathbb{R}^n$ are affine sets, then
  $S_1 \cap S_2$ is affine. To prove this, let
  $x_1, x_2 \in S_1 \cap S_2$, and let $\theta \in \mathbb{R}$. Then
  $x_1, x_2 \in S_i$ ($i = 1, 2$) in particular. Thus, since each of
  the sets $S_1$ and $S_2$ are affine,
  $\theta x_1 + (1 - \theta) x_2 \in S_i$ ($i = 1, 2$). Thus
  $\theta x_1 + (1 - \theta) x_2 \in S_1 \cap S_2$, so that
  $S_1 \cap S_2$ is affine.

  Now for the main result. Let $C \subseteq \mathbb{R}^n$ be an affine
  set, and let $L \subseteq \mathbb{R}^n$ be a line. Then $L$ is also
  an affine set. Thus $L \cap C$ is affine by the lemma above.

  Conversely, let $C \subseteq \mathbb{R}^n$ be an arbitrary set, and
  suppose that for every line $L \subseteq \mathbb{R}^n$, $L \cap C$
  is affine. Let $x_1, x_2 \in C$ and $\theta \in \mathbb{R}$. Let
  \begin{equation*}
    L = \{\alpha x_1 + (1 - \alpha) x_2 : \alpha \in \mathbb{R}\}
  \end{equation*}
  Then $L \cap C$ is affine by assumption. Thus since
  $x_1, x_2 \in L \cap C$,
  $\theta x_1 + (1 - \theta) x_2 \in L \cap C$. In particular,
  $\theta x_1 + (1 - \theta) x_2 \in C$, so that $C$ is affine.

\item Let $C \subseteq \mathbb{R}^n$ be closed and midpoint
  convex. Let $x_1, x_2 \in C$ and $0 \leq \theta \leq 1$. I claim
  that for every $n \in \mathbb{N}$, the set
  \begin{equation*}
    P_n = \left \{ \frac{m}{2^n} x_1 + \left ( 1 - \frac{m}{2^n} \right )
      : m \in \mathbb{Z}, 0 \leq m \leq 2^n \right \}
  \end{equation*}
  is contained in $C$. I will proceed by induction on $n$. When
  $n = 1$, we have
  \begin{equation*}
    P_1 = \left \{ x_1, \frac{1}{2} (x_1 + x_2), x_2 \right \} \subseteq C,
  \end{equation*}
  since $(x_1 + x_2) / 2 \in C$ since $C$ is midpoint convex. This
  establishes the base case. Now let $n > 1$ and suppose that
  $P_n \subseteq C$. Let $k \in \mathbb{Z}$, with
  $0 \leq k \leq 2^{n+1}$. If $k$ is of the form $k = 2m$ for some
  $m \in \mathbb{Z}$, $0 \leq m \leq 2^n$, then
  \begin{equation*}
    \frac{k}{2^{n+1}} x_1 + \left ( 1 - \frac{k}{2^{n+1}} \right ) x_2
      = \frac{m}{2^n} x_1 + \left ( 1 - \frac{m}{2^n} \right ) x_2 \in P_n.
  \end{equation*}
  So suppose $k$ is of the form $k = 2m + 1$ for some
  $m \in \mathbb{Z}$, $0 \leq m < 2^n$. Then by midpoint convexity of
  $C$,
  \begin{align*}
    % TODO: Less hacky alignment job...
    C \ni \frac{1}{2}& \left [ \frac{m}{2^n} x_1 +
      \left ( 1 - \frac{m}{2^n} \right ) x_2 \right ] +
    \frac{1}{2} \left [ \frac{m + 1}{2^n} x_1 +
      \left ( 1 - \frac{m + 1}{2^n} \right ) x_2 \right ] \\
    &= \frac{2m + 1}{2^{n+1}} x_1 +
         \left ( 1 - \frac{2m + 1}{2^{n+1}} \right ) x_2 \\
    &= \frac{k}{2^{n+1}} x_1 + \left ( 1 - \frac{k}{2^{n+1}} \right ) x_2.
  \end{align*}
  Thus $P_{n+1} \subseteq C$. Hence $P_n \subseteq C$ for every
  $n \in \mathbb{N}$ by induction.

  Next, define a sequence $\{y_n\}$ in $C$ as follows:
  \begin{equation*}
    y_n = \frac{m}{2^n} x_1 + \left ( 1 - \frac{m}{2^n} \right ) x_2
      \quad \text{for every $n \in \mathbb{N}$}
  \end{equation*}
  where
  \begin{equation*}
    m = \max \left \{k \in \mathbb{Z} :
      0 \leq k < 2^n, \frac{k}{2^n} \leq \theta \right \}.
  \end{equation*}
  Thus, for every $n \in \mathbb{N}$,
  \begin{equation*}
    \frac{m}{2^n} \leq \theta \leq \frac{m + 1}{2^n},
  \end{equation*}
  or equivalently (by rearranging this inequality),
  \begin{equation*}
    \left | \theta - \frac{m}{2^n} \right | \leq \frac{1}{2^n}.
  \end{equation*}
  Now let $y = \theta x_1 + (1 - \theta) x_2$. I claim that
  $y_n \to y$ as $n \to \infty$. We have
  \begin{align*}
    \lVert y_n - y \rVert_2
      &= \left \lVert \left [
           \frac{m}{2^n} x_1 + \left ( 1 - \frac{m}{2^n} \right ) x_2 \right ] -
           [ \theta x_1 + (1 - \theta) x_2 ] \right \rVert_2 \\
      &= \left \lVert \left ( \frac{m}{2^n} - \theta \right ) x_1 -
           \left ( \frac{m}{2^n} - \theta \right ) x_2 \right \rVert_2 \\
      &= \left | \frac{m}{2^n} - \theta \right | \lVert x_1 - x_2 \rVert_2 \\
      &\leq \frac{1}{2^n} \lVert x_1 - x_2 \rVert_2
        \to 0 \text{ as } n \to \infty.
  \end{align*}
  Thus $\lVert y_n - y \rVert_2 \to 0$ as $n \to \infty$, so that
  $y_n \to y$ as $n \to \infty$, as claimed. Thus, since $C$ is
  closed, $y \in C$. This shows that $C$ is convex.

\item
\begin{enumerate}
\item Let $x, v \in \mathbb{R}^n$ and define the line
  \begin{equation*}
    L = \{x + \theta v : \theta \in \mathbb{R}\}.
  \end{equation*}
  It suffices to show that $L \cap C$ is convex. Let
  $\theta \in \mathbb{R}$. Then
  \begin{align*}
    % TODO: Less hacky alignment job...
    (x +& \theta v)^T A (x + \theta v) + b^T (x + \theta v) + c \\
      &= x^T Ax + 2\theta x^T Av + \theta^2 v^T Av + b^T x + \theta b^T v + c \\
      &= \alpha \theta^2 + \beta \theta + \gamma,
  \end{align*}
  where
  \begin{align*}
    \alpha &= v^T Av \\
    \beta &= 2x^T Av + b^T v \\
    \gamma &= x^T Ax + b^T x + c.
  \end{align*}
  Thus we can write
  \begin{equation*}
    L \cap C = \{x + \theta v : \theta \in \mathbb{R},
      \alpha \theta^2 + \beta \theta + \gamma \leq 0\}.
  \end{equation*}

  Next, define the set
  \begin{equation*}
    S = \{\theta \in \mathbb{R}
      : \alpha \theta^2 + \beta \theta + \gamma \leq 0\}.
  \end{equation*}
  Since $A \succeq 0$, $\alpha = v^T Av \geq 0$, and so
  $\alpha \theta^2 + \beta \theta + \gamma$ is a (possibly degenerate)
  upward-curving parabola. If the parabola has no real roots, then
  $S = \emptyset$, which is trivially convex. If $\alpha \neq 0$, then
  it has roots $\theta_1 \leq \theta_2$ (not necessarily distinct),
  and $S = [\theta_1, \theta_2]$, which is a convex set. If
  $\alpha = 0$ (i.e., the degenerate case), then $S$ reduces to
  \begin{equation*}
    S = \{\theta \in \mathbb{R} : \beta \theta + \gamma \in (-\infty, 0]\},
  \end{equation*}
  i.e., the preimage of the convex set $(-\infty, 0]$ under the affine
  function $\theta \mapsto \beta \theta + \gamma$, and thus is convex.

  Finally, define an affine function $f : \mathbb{R} \to \mathbb{R}^n$ by
  $f(\theta) = x + \theta v$. Then we can write $L \cap C =
  f(S)$. This shows that $L \cap C$ is convex, and therefore $C$ is
  convex.

  \bigskip

  The converse, however, does not hold. To see this, consider the set
  \begin{equation*}
    C = \{x \in \mathbb{R} : -x^2 - 1 \leq 0\}
      = \{x \in \mathbb{R} : x^2 + 1 \geq 0\}
      = \mathbb{R},
  \end{equation*}
  which is convex, but $A = -1 < 0$.
\end{enumerate}

\item
\begin{enumerate}
\item A slab can be expressed as the intersection of two halfspaces:
  \begin{equation*}
    \{x \in \mathbb{R}^n : \alpha \leq a^T x \leq \beta\}
      = \{x \in \mathbb{R}^n : a^T x \geq \alpha\}
          \cap \{x \in \mathbb{R}^n : a^T x \leq \beta\}.
  \end{equation*}
  Thus, since halfspaces are convex and set intersection preserves
  convexity, a slab is convex.

\item Let
  \begin{equation*}
    R = \{x \in \mathbb{R}^n
      : \alpha_i \leq x_i \leq \beta_i, 1 \leq i \leq n\}
  \end{equation*}
  denote a rectangle. Let $x, y \in R$ and $0 \leq \theta \leq
  1$. Then for every $1 \leq i \leq n$,
  \begin{align*}
    \theta x_i + (1 - \theta) y_i
      &\geq \theta \alpha_i + (1 - \theta) \alpha_i
      = \alpha_i \\
    \theta x_i + (1 - \theta) y_i
      &\leq \theta \beta_i + (1 - \theta) \beta_i
      = \beta_i.
  \end{align*}
  Thus $\theta x + (1 - \theta) y \in R$, so that $R$ is convex.

\item A wedge can be expressed as the intersection of two halfspaces:
  \begin{equation*}
    \{x \in \mathbb{R}^n : a_1^T x \leq b_1, a_2^T x \leq b_2\}
      = \{x \in \mathbb{R}^n : a_1^T x \leq b_1\}
          \cap \{x \in \mathbb{R}^n : a_2^T x \leq b_2\}
  \end{equation*}
  Thus, since halfspaces are convex and set intersection preserves
  convexity, a wedge is convex.

\item This set can be expressed as the intersection
  \begin{equation*}
    \{x \in \mathbb{R}^n : \lVert x - x_0 \rVert_2 \leq \lVert x - y \rVert_2
      \text{ for every $y \in S$}\}
    = \bigcap_{y \in S} A_y,
  \end{equation*}
  where for every $y \in S$, the set $A_y$ is defined as
  \begin{equation*}
  \end{equation*}

\end{enumerate}

\end{enumerate}

\end{document}

\message{ !name(hw1.tex) !offset(-256) }
