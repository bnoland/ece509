\documentclass[letterpaper,12pt]{article}
\usepackage[utf8]{inputenc}
\usepackage[top=1.25in, bottom=1.25in, left=1in, right=1in]{geometry}
\usepackage{amsmath}
\usepackage{amsfonts}
\usepackage{enumitem}

\setenumerate{parsep=0em, listparindent=\parindent}

\DeclareMathOperator{\Tr}{Tr}
\DeclareMathOperator{\dom}{dom}
\DeclareMathOperator{\diag}{diag}

\title{Homework 1}
\author{Benjamin Noland}
\date{}

\begin{document}

\maketitle

\begin{enumerate}
\item (Boyd \& Vandenberghe, Exercise 2.2) Let
  $C \subseteq \mathbb{R}^n$ be a convex set, and let
  $L \subseteq \mathbb{R}^n$ be a line. Then $L$ is also a convex
  set. Since set intersection preserves convexity, $L \cap C$ is
  therefore a convex set.

  Conversely, let $C \subset \mathbb{R}^n$ be an arbitrary set, and
  suppose that for every line $L \subseteq \mathbb{R}^n$, $L \cap C$
  is convex. Let $x_1, x_2 \in C$ and $0 \leq \theta \leq 1$. Let
  \begin{equation*}
    L = \{\alpha x_1 + (1 - \alpha) x_2 : \alpha \in \mathbb{R}\}.
  \end{equation*}
  Then $L \cap C$ is convex by assumption. Thus, since
  $x_1, x_2 \in L \cap C$,
  $\theta x_1 + (1 - \theta) x_2 \in L \cap C$, and so
  $\theta x_1 + (1 - \theta) x_2 \in C$ in particular. Thus $C$ is
  convex.

  \bigskip

  The corresponding result for affine sets requires a short lemma: if
  $S_1, S_2 \subseteq \mathbb{R}^n$ are affine sets, then
  $S_1 \cap S_2$ is affine. To prove this, let
  $x_1, x_2 \in S_1 \cap S_2$, and let $\theta \in \mathbb{R}$. Then
  $x_1, x_2 \in S_i$ ($i = 1, 2$) in particular. Thus, since each of
  the sets $S_1$ and $S_2$ are affine,
  $\theta x_1 + (1 - \theta) x_2 \in S_i$ ($i = 1, 2$). Thus
  $\theta x_1 + (1 - \theta) x_2 \in S_1 \cap S_2$, so that
  $S_1 \cap S_2$ is affine.

  Now for the main result. Let $C \subseteq \mathbb{R}^n$ be an affine
  set, and let $L \subseteq \mathbb{R}^n$ be a line. Then $L$ is also
  an affine set. Thus $L \cap C$ is affine by the lemma above.

  Conversely, let $C \subseteq \mathbb{R}^n$ be an arbitrary set, and
  suppose that for every line $L \subseteq \mathbb{R}^n$, $L \cap C$
  is affine. Let $x_1, x_2 \in C$ and $\theta \in \mathbb{R}$. Let
  \begin{equation*}
    L = \{\alpha x_1 + (1 - \alpha) x_2 : \alpha \in \mathbb{R}\}
  \end{equation*}
  Then $L \cap C$ is affine by assumption. Thus since
  $x_1, x_2 \in L \cap C$,
  $\theta x_1 + (1 - \theta) x_2 \in L \cap C$. In particular,
  $\theta x_1 + (1 - \theta) x_2 \in C$, so that $C$ is affine.

\item (Boyd \& Vandenberghe, Exercise 2.3) Let
  $C \subseteq \mathbb{R}^n$ be closed and midpoint convex. Let
  $x_1, x_2 \in C$ and $0 \leq \theta \leq 1$. I claim that for every
  $n \in \mathbb{N}$, the set
  \begin{equation*}
    P_n = \left \{ \frac{m}{2^n} x_1 + \left ( 1 - \frac{m}{2^n} \right ) :
      m \in \mathbb{Z}, 0 \leq m \leq 2^n \right \}
  \end{equation*}
  is contained in $C$. I will proceed by induction on $n$. When
  $n = 1$, we have
  \begin{equation*}
    P_1 = \left \{ x_1, \frac{1}{2} (x_1 + x_2), x_2 \right \} \subseteq C,
  \end{equation*}
  since $(x_1 + x_2) / 2 \in C$ because $C$ is midpoint convex. This
  establishes the base case. Now let $n > 1$ and suppose that
  $P_n \subseteq C$. Let $k \in \mathbb{Z}$, with
  $0 \leq k \leq 2^{n+1}$. If $k$ is of the form $k = 2m$ for some
  $m \in \mathbb{Z}$, $0 \leq m \leq 2^n$, then
  \begin{equation*}
    \frac{k}{2^{n+1}} x_1 + \left ( 1 - \frac{k}{2^{n+1}} \right ) x_2
      = \frac{m}{2^n} x_1 + \left ( 1 - \frac{m}{2^n} \right ) x_2 \in P_n.
  \end{equation*}
  So suppose $k$ is of the form $k = 2m + 1$ for some
  $m \in \mathbb{Z}$, $0 \leq m < 2^n$. Then by midpoint convexity of
  $C$,
  \begin{align*}
    % TODO: Less hacky alignment job...
    C \ni \frac{1}{2}& \left [ \frac{m}{2^n} x_1 +
      \left ( 1 - \frac{m}{2^n} \right ) x_2 \right ] +
    \frac{1}{2} \left [ \frac{m + 1}{2^n} x_1 +
      \left ( 1 - \frac{m + 1}{2^n} \right ) x_2 \right ] \\
    &= \frac{2m + 1}{2^{n+1}} x_1 +
         \left ( 1 - \frac{2m + 1}{2^{n+1}} \right ) x_2 \\
    &= \frac{k}{2^{n+1}} x_1 + \left ( 1 - \frac{k}{2^{n+1}} \right ) x_2.
  \end{align*}
  Thus $P_{n+1} \subseteq C$. Hence $P_n \subseteq C$ for every
  $n \in \mathbb{N}$ by induction.

  Next, define a sequence $\{y_n\}$ in $C$ as follows:
  \begin{equation*}
    y_n = \frac{m}{2^n} x_1 + \left ( 1 - \frac{m}{2^n} \right ) x_2
      \quad \text{for every $n \in \mathbb{N}$}
  \end{equation*}
  where
  \begin{equation*}
    m = \max \left \{k \in \mathbb{Z} :
      0 \leq k < 2^n, \frac{k}{2^n} \leq \theta \right \}.
  \end{equation*}
  Thus, for every $n \in \mathbb{N}$,
  \begin{equation*}
    \frac{m}{2^n} \leq \theta \leq \frac{m + 1}{2^n},
  \end{equation*}
  or equivalently (by rearranging this inequality),
  \begin{equation*}
    \left | \theta - \frac{m}{2^n} \right | \leq \frac{1}{2^n}.
  \end{equation*}
  Now let $y = \theta x_1 + (1 - \theta) x_2$. I claim that
  $y_n \to y$ as $n \to \infty$. We have
  \begin{align*}
    \lVert y_n - y \rVert_2
      &= \left \lVert \left [
           \frac{m}{2^n} x_1 + \left ( 1 - \frac{m}{2^n} \right ) x_2 \right ] -
           [ \theta x_1 + (1 - \theta) x_2 ] \right \rVert_2 \\
      &= \left \lVert \left ( \frac{m}{2^n} - \theta \right ) x_1 -
           \left ( \frac{m}{2^n} - \theta \right ) x_2 \right \rVert_2 \\
      &= \left | \frac{m}{2^n} - \theta \right | \lVert x_1 - x_2 \rVert_2 \\
      &\leq \frac{1}{2^n} \lVert x_1 - x_2 \rVert_2
        \to 0 \text{ as } n \to \infty.
  \end{align*}
  Thus $\lVert y_n - y \rVert_2 \to 0$ as $n \to \infty$, so that
  $y_n \to y$ as $n \to \infty$, as claimed. Thus, since $C$ is
  closed, $y \in C$. This shows that $C$ is convex.

\item (Boyd \& Vandenberghe, Exercise 2.10)
\begin{enumerate}
\item Let $x, v \in \mathbb{R}^n$ and define the line
  \begin{equation*}
    L = \{x + \theta v : \theta \in \mathbb{R}\}.
  \end{equation*}
  It suffices to show that $L \cap C$ is convex. Let
  $\theta \in \mathbb{R}$. Then
  \begin{align*}
    % TODO: Less hacky alignment job...
    (x +& \theta v)^T A (x + \theta v) + b^T (x + \theta v) + c \\
      &= x^T Ax + 2\theta x^T Av + \theta^2 v^T Av + b^T x + \theta b^T v + c \\
      &= \alpha \theta^2 + \beta \theta + \gamma,
  \end{align*}
  where
  \begin{align*}
    \alpha &= v^T Av \\
    \beta &= 2x^T Av + b^T v \\
    \gamma &= x^T Ax + b^T x + c.
  \end{align*}
  Thus we can write
  \begin{equation*}
    L \cap C = \{x + \theta v : \theta \in \mathbb{R},
      \alpha \theta^2 + \beta \theta + \gamma \leq 0\}.
  \end{equation*}

  Next, define the set
  \begin{equation*}
    S = \{\theta \in \mathbb{R} :
      \alpha \theta^2 + \beta \theta + \gamma \leq 0\}.
  \end{equation*}
  Since $A \succeq 0$, $\alpha = v^T Av \geq 0$, and so
  $\alpha \theta^2 + \beta \theta + \gamma$ is a (possibly degenerate)
  upward-curving parabola. If the parabola has no real roots, then
  $S = \emptyset$, which is trivially convex. If $\alpha \neq 0$, then
  it has roots $\theta_1 \leq \theta_2$ (not necessarily distinct),
  and $S = [\theta_1, \theta_2]$, which is a convex set. If
  $\alpha = 0$ (i.e., the degenerate case), then $S$ reduces to
  \begin{equation*}
    S = \{\theta \in \mathbb{R} : \beta \theta + \gamma \in (-\infty, 0]\},
  \end{equation*}
  i.e., the preimage of the convex set $(-\infty, 0]$ under the affine
  function $\theta \mapsto \beta \theta + \gamma$, and thus is convex.

  Finally, define an affine function $f : \mathbb{R} \to \mathbb{R}^n$ by
  $f(\theta) = x + \theta v$. Then we can write $L \cap C =
  f(S)$. This shows that $L \cap C$ is convex, and therefore $C$ is
  convex.

  \bigskip

  The converse, however, does not hold. To see this, consider the set
  \begin{equation*}
    C = \{x \in \mathbb{R} : -x^2 - 1 \leq 0\}
      = \{x \in \mathbb{R} : x^2 + 1 \geq 0\}
      = \mathbb{R},
  \end{equation*}
  which is convex, but $A = -1 < 0$.
\end{enumerate}

\item (Boyd \& Vandenberghe, Exercise 2.12)
\begin{enumerate}
\item A slab can be expressed as the intersection of two halfspaces:
  \begin{equation*}
    \{x \in \mathbb{R}^n : \alpha \leq a^T x \leq \beta\}
      = \{x \in \mathbb{R}^n : a^T x \geq \alpha\}
          \cap \{x \in \mathbb{R}^n : a^T x \leq \beta\}.
  \end{equation*}
  Thus, since halfspaces are convex and set intersection preserves
  convexity, a slab is convex.

\item Let
  \begin{equation*}
    R = \{x \in \mathbb{R}^n :
      \alpha_i \leq x_i \leq \beta_i, 1 \leq i \leq n\}
  \end{equation*}
  denote a rectangle. Let $x, y \in R$ and $0 \leq \theta \leq
  1$. Then for every $1 \leq i \leq n$,
  \begin{align*}
    \theta x_i + (1 - \theta) y_i
      &\geq \theta \alpha_i + (1 - \theta) \alpha_i
      = \alpha_i \\
    \theta x_i + (1 - \theta) y_i
      &\leq \theta \beta_i + (1 - \theta) \beta_i
      = \beta_i.
  \end{align*}
  Thus $\theta x + (1 - \theta) y \in R$, so that $R$ is convex.

\item A wedge can be expressed as the intersection of two halfspaces:
  \begin{equation*}
    \{x \in \mathbb{R}^n : a_1^T x \leq b_1, a_2^T x \leq b_2\}
      = \{x \in \mathbb{R}^n : a_1^T x \leq b_1\}
          \cap \{x \in \mathbb{R}^n : a_2^T x \leq b_2\}
  \end{equation*}
  Thus, since halfspaces are convex and set intersection preserves
  convexity, a wedge is convex.

\item This set can be expressed as the intersection
  \begin{equation*}
    C = \{x \in \mathbb{R}^n : \lVert x - x_0 \rVert_2
          \leq \lVert x - y \rVert_2 \text{ for every $y \in S$}\}
    = \bigcap_{y \in S} C_y,
  \end{equation*}
  where for every $y \in S$, the set $C_y$ is defined as
  \begin{equation*}
    C_y = \{x \in \mathbb{R}^n :
      \lVert x - x_0 \rVert_2 \leq \lVert x - y \rVert_2\}.
  \end{equation*}

  Let $y \in S$. Then $x \in C_y$ if and only if
  \begin{equation*}
    (x - x_0)^T (x - x_0) \leq (x - y)^T (x - y),
  \end{equation*}
  or equivalently (after some algebra),
  \begin{equation*}
    (2y - x_0)^T x \leq y^T y - x_0^T x_0.
  \end{equation*}
  Thus
  \begin{equation*}
    C_y = \{x \in \mathbb{R}^n : (2y - x_0)^T x \leq y^T y - x_0^T x_0\}.
  \end{equation*}
  If $2y - x_0 \neq 0$, then $C_y$ is a halfspace, hence convex. If
  $2y - x_0 = 0$, then $x_0^T x_0 = 4y^T y$, and so $C_y = \emptyset$
  if $y = 0$, and $C_y = \mathbb{R}$ if $y \neq 0$. In either case,
  $C_y$ is convex.

  Therefore, since $C_y$ is convex for every $y \in S$, the
  intersection $C = \bigcap_{y \in S} C_y$ is also convex.
\end{enumerate}

\item (Boyd \& Vandenberghe, Exercise 2.21) Let
  $A \subseteq \mathbb{R}^{n+1}$ denote the set in question. Let
  $(a_1, b_1), (a_2, b_2) \in A$, and let $\theta_1, \theta_2 \geq
  0$. Then
  \begin{equation*}
    \theta_1 (a_1, b_1) + \theta_2 (a_2, b_2)
      = (\theta_1 a_1 + \theta_2 a_2, \theta_1 b_1 + \theta b_2).
  \end{equation*}
  Thus, for every $x \in C$,
  \begin{equation*}
    (\theta_1 a_1 + \theta_2 a_2)^T x
      = \theta_1 a_1^T x + \theta_2 a_2^T x
      \leq \theta_1 b_1 + \theta_2 b_2,
  \end{equation*}
  and for every $x \in D$,
  \begin{equation*}
    (\theta_1 a_1 + \theta_2 a_2)^T x
      = \theta_1 a_1^T x + \theta_2 a_2^T x
      \geq \theta_1 b_1 + \theta_2 b_2.
  \end{equation*}
  Thus $\theta_1 (a_1, b_1) + \theta_2 (a_2, b_2) \in A$, so that $A$
  is a convex cone. In particular, if there is no hyperplane
  separating $C$ and $D$, then for every
  $(a, b) \in \mathbb{R}^{n+1}$ with
  \begin{align*}
    a^T x &\leq b \quad \text{for every $x \in C$} \\
    a^T x &\geq b \quad \text{for every $x \in D$}
  \end{align*}
  we must have $a = 0$, since otherwise
  $\{x \in \mathbb{R}^n : a^T x = b\}$ would be a hyperplane
  separating $C$ and $D$. Hence $0 \leq b$ and $0 \geq b$, so that
  $b = 0$. It therefore follows that $A$ is the singleton $A = \{0\}$
  in this case.

\item (Convexity of $x \mapsto e^{ax}$ on $\mathbb{R}$) First, note
  that the domain $\mathbb{R}$ is convex. In addition, since
  $x \mapsto e^{ax}$ is twice differentiable, we can compute
  \begin{equation*}
    \frac{d^2}{dx^2} e^{ax} = a^2 e^{ax} \geq 0
    \quad \text{for any $x \in \mathbb{R}$}.
  \end{equation*}
  Thus $x \mapsto e^{ax}$ is convex.

\item (Concavity of $x \mapsto \log x$ on $\mathbb{R}_{++}$) First,
  note that the domain $\mathbb{R}_{++} = (0, \infty)$, being an
  interval in $\mathbb{R}$, is convex. In addition, since
  $x \mapsto \log x$ is twice differentiable, we can compute
  \begin{equation*}
    \frac{d^2}{dx^2} \log x = -\frac{1}{x^2} \leq 0
    \quad \text{for any $x \in \mathbb{R}_{++}$}.
  \end{equation*}
  Thus $x \mapsto \log x$ is concave.

\item (Convexity of $X \mapsto -\log \det X$ on $S^n_{++}$) Let
  $f : S^n \to \mathbb{R}$, with $\dom f = S^n_{++}$, be defined by
  $f(X) = \log \det X$. Let $L$ be a line with
  $L \cap S^n_{++} \neq \emptyset$, say
  \begin{equation*}
    L = \{Z + tV : t \in \mathbb{R}\},
  \end{equation*}
  where $Z, V \in S^n$. Let $g : \mathbb{R} \to \mathbb{R}$, with
  \begin{equation*}
    \dom g = L \cap S^n_{++} = \{t \in \mathbb{R} : Z + tV \succ 0\},
  \end{equation*}
  denote the restriction of $f$ to $L$; that is, $g(t) = f(Z +
  tV)$. It suffices to show that $g$ is a convex function. We can
  assume without loss of generality that $Z \succ 0$, since because
  $\dom g \neq \emptyset$, we can always choose a basepoint
  $Z' = Z + t' V \in S^n_{++}$ for some $t' \in \dom g$. Note that
  $\dom g$ is simply the preimage of $S^n_{++}$ under the continuous
  affine function $t \mapsto Z + tV$. Thus, since $S^n_{++}$ is an
  open subset of $S^n$, $\dom g$ is an open, convex subset of
  $\mathbb{R}$, hence an open interval.

  Since $Z \succ 0$, $Z$ has a unique matrix square root
  $Z^{1/2}$. Thus we can write
  \begin{align*}
    g(t) &= \log \det(Z + tV) \\
      &= \log \det[Z^{1/2}(I + tZ^{-1/2}VZ^{-1/2})Z^{1/2}] \\
      &= \log [\det(Z^{1/2}) \det(I + tZ^{-1/2}VZ^{-1/2}) \det(Z^{1/2})] \\
      &= \log [\det(Z) \det(I + tZ^{-1/2}VZ^{-1/2})] \\
      &= \log \det Z + \log \det(I + tZ^{-1/2}VZ^{-1/2}).
  \end{align*}
  Note that
  \begin{align*}
    \det(I + tZ^{-1/2}VZ^{-1/2})
      &= t^n \det \left ( \frac{1}{t} - (-Z^{-1/2}VZ^{-1/2}) \right ) \\
      &= t^n \prod_{i=1}^n \left ( \frac{1}{t} - (- \lambda_i) \right ) \\
      &= \prod_{i=1}^n (1 + t\lambda_i),
  \end{align*}
  where $\lambda_1, \ldots, \lambda_n$ are the eigenvalues of
  $Z^{-1/2}VZ^{-1/2}$. Thus,
  \begin{align*}
    g(t) &= \log \det Z + \log \det(I + tZ^{-1/2}VZ^{-1/2}) \\
      &= \log \det Z + \log \prod_{i=1}^n (1 + t\lambda_i) \\
      &= \log \det Z + \sum_{i=1}^n \log(1 + t\lambda_i).
  \end{align*}
  Taking derivates, we find that
  \begin{equation*}
    g'(t) = \sum_{i=1}^n \frac{\lambda_i}{1 + t\lambda_i}
    \quad \text{and} \quad
    g''(t) = \sum_{i=1}^n -\frac{\lambda_i^2}{(1 + t\lambda_i)^2} \leq 0
  \end{equation*}
  for every $t \in \dom g$. Since $\dom g$ is open, we therefore
  conclude that $g$, and hence $f$, is concave. In particular,
  $-\log \det X$ is convex on $S^n_{++}$.

\item (Boyd \& Vandenberghe, Exercise 3.11) Since $f$ is convex and
  differentiable, we have
  \begin{align*}
    f(y) &\geq f(x) + \nabla f(x)^T (y - x) \\
    f(x) &\geq f(y) + \nabla f(y)^T (x - y)
  \end{align*}
  for every $x, y \in \dom f$. Thus, negating both sides of the second
  inequality, we find
  \begin{equation*}
    -f(x) \leq -f(y) + \nabla f(y)^T (y - x),
  \end{equation*}
  and adding this to the first inequality above, we get
  \begin{equation*}
    \nabla f(y)^T (y - x) \geq \nabla f(x)^T (y - x),
  \end{equation*}
  or equivalently,
  \begin{equation*}
    (\nabla f(y) - \nabla f(x))^T (y - x) \geq 0.
  \end{equation*}
  Therefore $\nabla f$ is monotone.

  \bigskip

  The converse, however, does not hold. As an example, consider the
  function $\psi : \mathbb{R}^2 \to \mathbb{R}$ defined by
  \begin{equation*}
    \psi(x)
      = \begin{pmatrix}
        x_1 \\
        x_1 + x_2
      \end{pmatrix}
      = \begin{pmatrix}
        1 & 0 \\
        1 & 1
      \end{pmatrix}
      \begin{pmatrix}
        x_1 \\
        x_2
      \end{pmatrix}.
  \end{equation*}
  Note that for any $x \in \mathbb{R}^2$,
  \begin{align*}
    x^T
    \begin{pmatrix}
      1 & 0 \\
      1 & 1
    \end{pmatrix}
    x
    &= x_1^2 + x_2^2 + x_1 x_2 \\
    &= x_1^2 + x_2^2 + \frac{1}{2} x_1 x_2 + \frac{1}{2} x_1 x_2 \\
    &= x^T
    \begin{pmatrix}
      1 & 1/2 \\
      1/2 & 1
    \end{pmatrix}
    x \geq 0,
  \end{align*}
  since the matrix in the last inequality is positive definite (this
  can be easily verified using Sylvester's criterion). It follows
  immediately that $\psi$ is monotone, since for any
  $x, y \in \mathbb{R}^2$,
  \begin{equation*}
    (\psi(x) - \psi(y))^T (x - y)
    = (x - y)^T
    \begin{pmatrix}
      1 & 1/2 \\
      1/2 & 1
    \end{pmatrix}
    (x - y) \geq 0.
  \end{equation*}
  But if we had $\psi = \nabla f$ for some differentiable function
  $f : \mathbb{R}^2 \to \mathbb{R}$, we would necessarily have
  \begin{equation*}
    \frac{\partial^2 f}{\partial x_1 \partial x_2}
      = \frac{\partial^2 f}{\partial x_2 \partial x_1} = 0
    \quad \text{and} \quad
    \frac{\partial^2 f}{\partial x_1 \partial x_2} = 1,
  \end{equation*}
  which is impossible.

\item (Boyd \& Vandenberghe, Exercise 3.18)
\begin{enumerate}
\item As in problem 8, let $g : \mathbb{R} \to \mathbb{R}$ denote the
  restriction of $f$ to the line
  \begin{equation*}
    L = \{Z + tV : t \in \mathbb{R}\},
  \end{equation*}
  where as before we can assume without loss of generality that
  $Z \succ 0$. Thus $Z$ has a unique matrix square root
  $Z^{1/2}$. Furthermore, the matrix $Z^{-1/2}VZ^{-1/2}$ is real
  symmetric, and can therefore be written as
  $Z^{-1/2}VZ^{-1/2} = Q^T \Lambda Q$, where $Q$ is an orthogonal
  matrix and $\Lambda = \diag\{\lambda_1, \ldots, \lambda_n\}$, where
  $\lambda_1, \ldots, \lambda_n$ are the eigenvalues of
  $Z^{-1/2}VZ^{-1/2}$. We can therefore write
  \begin{align*}
    g(t) &= f(Z + tV) \\
      &= \Tr[(Z + tV)^{-1}] \\
      &= \Tr[Z^{-1/2}(I + tZ^{-1/2}VZ^{-1/2})^{-1}Z^{-1/2}] \\
      &= \Tr[Z^{-1}(I + tZ^{-1/2}VZ^{-1/2})^{-1}] \\
      &= \Tr[Z^{-1}(I + tQ^T \Lambda Q)^{-1}] \\
      &= \Tr[Z^{-1}(Q^T + t\Lambda Q)^{-1} Q] \\
      &= \Tr[Z^{-1}Q^T (I + t\Lambda)^{-1} Q] \\
      &= \Tr[Q Z^{-1}Q^T (I + t\Lambda)^{-1}] \\
      &= \sum_{i=1}^n [Q Z^{-1} Q^T]_{ii} \frac{1}{1 + t\lambda_i}.
  \end{align*}
  Each of the functions $t \mapsto 1 / (1 + t\lambda_i)$ is convex on
  $\dom g$, and it therefore follows that $g$ itself is convex. Thus
  $f$ is convex.

\item As in problem 8, let $g : \mathbb{R} \to \mathbb{R}$ denote the
  restriction of $f$ to the line
  \begin{equation*}
    L = \{Z + tV : t \in \mathbb{R}\},
  \end{equation*}
  where as before we can assume without loss of generality that
  $Z \succ 0$. Thus $Z$ has a unique matrix square root
  $Z^{1/2}$. We can therefore write
  \begin{align*}
    g(t) &= f(Z + tV) \\
      &= [\det(Z + tV)]^{1/n} \\
      &= [\det[Z^{1/2}(I + tZ^{-1/2}VZ^{-1/2})Z^{1/2}]]^{1/n} \\
      &= [\det Z \det(I + tZ^{-1/2}VZ^{-1/2})]^{1/n} \\
      &= (\det Z)^{1/n} \left [ \prod_{i=1}^n (1 + t\lambda_i) \right ]^{1/n},
  \end{align*}
  where $\lambda_1, \ldots, \lambda_n$ are the eigenvalues of
  $Z^{-1/2}VZ^{-1/2}$. Since $Z \succ 0$, $\det Z > 0$. In addition,
  the geometric mean function $x \mapsto (\prod_{i=1}^n x_i)^{1/n}$ is
  concave on $\mathbb{R}^n_{++}$. It therefore follows that $g$ is
  concave. Thus $f$ is concave as well.

\end{enumerate}

\end{enumerate}

\end{document}
