\documentclass[letterpaper,12pt]{article}
\usepackage[utf8]{inputenc}
\usepackage[top=1.25in, bottom=1.25in, left=1in, right=1in]{geometry}
\usepackage{amsmath}
\usepackage{amsfonts}
\usepackage{enumitem}

\setenumerate{parsep=0em, listparindent=\parindent}

\DeclareMathOperator{\Tr}{Tr}
\DeclareMathOperator{\dom}{dom}
\DeclareMathOperator{\diag}{diag}
\DeclareMathOperator{\dist}{dist}

\title{Homework 1}
\author{Benjamin Noland}
\date{}

\begin{document}

\maketitle

\begin{enumerate}
\item (Boyd \& Vandenberghe, Exercise 2.12)
\begin{enumerate}
\setcounter{enumii}{4}
\item This set is not necessarily convex. As a counterexample, define
  the sets $S$ and $T$ by $S = (-\infty, -2] \cup [2, \infty)$ and
  $T = [-1, 1]$. Then the set
  \begin{equation*}
    \{x \in \mathbb{R} : \dist(x, S) \leq \dist(x, T)\}
      = (-\infty, -3/2] \cup [3/2, \infty)
  \end{equation*}
  is not convex.

\item For every $s \in S_2$, define the set
  \begin{equation*}
    C_s = \{x \in \mathbb{R}^n : x + s \in S_1\}.
  \end{equation*}
  To see that these sets are convex, let $s \in S_2$, and let
  $x_1, x_2 \in C_s$, and $0 \leq \theta \leq 1$. Then since
  $x_1 + s, x_2 + s \in S_1$, convexity of $S_1$ implies that
  \begin{equation*}
    \theta x_1 + (1 - \theta) x_2 + s
      = \theta (x_1 + s) + (1 - \theta)(x_2 + s) \in S_2.
  \end{equation*}
  Thus $C_s$ is convex, as claimed. Therefore
  \begin{equation*}
    \{x \in \mathbb{R}^n : x + S_2 \subseteq S_1\} = \bigcap_{s \in S_2} C_s
  \end{equation*}
  is convex, since set intersection preserves convexity.

\item Let
  \begin{equation*}
    C = \{x \in \mathbb{R}^n :
      \lVert x - a \rVert_2 \leq \theta \lVert x - b \rVert_2\}.
  \end{equation*}
  Then $x \in C$ if and only if
  \begin{equation*}
    (x - a)^T (x - a) \leq \theta^2 (x - b)^T (x - b),
  \end{equation*}
  or equivalently (after rearrangement),
  \begin{equation*}
    (1 - \theta^2) x^T x + 2(\theta^2 b - a)^T x
      + a^T a - \theta^2 b^T b \leq 0.
  \end{equation*}
  Thus, letting $A = (1 - \theta^2)I$, $\beta = 2(\theta^2 b - a)$,
  and $\gamma = a^T a - \theta^2 b^T b$, the set in question becomes
  \begin{equation*}
    C = \{x \in \mathbb{R}^n : x^T Ax + \beta^T x + \gamma \leq 0\}.
  \end{equation*}
  Thus, since $A \succeq 0$, Exercise~2.10~(a) implies that $C$ is
  convex.
\end{enumerate}

\item (Boyd \& Vandenberghe, Exercise 3.20)
\begin{enumerate}
\setcounter{enumii}{2}
\item Recall from Exercise~3.18~(a) that the map $g(X) = \Tr(X^{-1})$
  on $S^m_{++}$ is convex. In addition, the map
  $h(x) = A_0 + x_1 A_1 + \cdots + x_n A_n$ on $\mathbb{R}^n$ is
  affine. The map $f$ in question is the composition $f = g \circ h$,
  with domain $\dom f = \{x \in \mathbb{R}^n : h(x) \in \dom g\}$. A
  composition of this form is convex, and so $f$ is convex.
\end{enumerate}

\item (Boyd \& Vandenberghe, Exercise 3.21)
\begin{enumerate}
\item For every $1 \leq i \leq k$, define the map
  $f_i(x) = \lVert A^{(i)}x - b^{(i)} \rVert$ on $\mathbb{R}^m$. Since
  norms are convex, and each of the maps $f_i$ is the composition of a
  norm with an affine map, we see that each $f_i$ is convex. Since the
  map $f$ in question is the pointwise maximum of $f_1, \ldots, f_k$,
  we conclude that $f$ itself is convex.
\end{enumerate}

\end{enumerate}

\end{document}
